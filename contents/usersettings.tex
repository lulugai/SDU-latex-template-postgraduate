% !Mode:: "TeX:UTF-8"
% usersettings.tex
\usepackage{paralist}
\usepackage{dtk-logos}
\usepackage[table]{xcolor}
\usepackage{listings}
\usepackage{times} %times new roman
% Add Chapter to TOC && Remove Contents from TOC
\usepackage[chapter, nottoc]{tocbibind}
% 允许使用子图 subfloat
\usepackage{subfig}
% 允许使用图文混排
\usepackage{wrapfig}
% picinpar 和 picins 都无法成功实现图文混排。
% \usepackage{picinpar}
% \usepackage{picins}
% 使用 ccmap,则英文和数字都正常。如果论文重复率高的话,注释掉 ccmap ,英文和数字被映射为汉字,相当于增加干扰码,降低重复率。
% \usepackage{ccmap}

% \newtheorem{环境名}[编号延续]{显示名}[编号层次]
% 在下例中,我们定义了四个环境:定义、定理、引理和推论,它们都在一个section内编号,而引理和推论会延续定理的编号。
% \newtheorem{definition}{定义}[section]
% \newtheorem{theorem}{定理}[section]
% \newtheorem{lemma}[theorem]{引理}
% \newtheorem{corollary}[theorem]{推论}

\lstset{language=TeX}
% \lstset{extendedchars=false}
\lstset{breaklines}	
\lstset{numbers=left,numberstyle=\tiny,%commentstyle=\color{red!50!green!50!blue!50},
frame=shadowbox,rulesepcolor=\color{red!20!green!20!blue!20},escapeinside=``,
xleftmargin=2em,xrightmargin=2em,aboveskip=1em,backgroundcolor=\color{red!3!green!3!blue!3},
basicstyle=\small\ttfamily,stringstyle=\color{purple},keywordstyle=\color{blue!80}\bfseries,
commentstyle=\color{olive}}
\numberwithin{footnote}{page}
\renewcommand{\thefootnote}{\arabic{footnote}}
\renewcommand{\CTeX}{\SHUANG{C}\TeX}

\usepackage{multirow} % Required for multirows
\usepackage{caption}
\usepackage{etoolbox} % 导入 etoolbox 宏包
\usepackage{enumitem}
\setlist[enumerate]{topsep=0pt,partopsep=0pt,itemsep=0pt,parsep=0pt} %设置item间距

\AtBeginEnvironment{tabular}{\footnotesize} % 设置表格字体大小为 10pt
\AtBeginEnvironment{table}{\footnotesize} % 设置表格标题字体大小为 10pt
\captionsetup[table]{font=footnotesize,skip=6pt} % 设置表格标题字体大小为 10pt
\captionsetup[figure]{font=footnotesize,skip=6pt} % 设置图片标题字体大小为 10pt
\AtBeginEnvironment{figure}{\fontsize{10pt}{10pt}\selectfont} % 设置图片字体大小为 10pt,行距为 10pt

\captionsetup{labelsep=space}
\renewcommand{\thefigure}{\arabic{chapter}-\arabic{figure}}
\renewcommand{\thetable}{\arabic{chapter}-\arabic{table}}
\renewcommand{\theequation}{\arabic{chapter}-\arabic{equation}}

%设置参考文献标号间距
\makeatletter
\renewcommand{\@biblabel}[1]{[#1]
   \ifnum#1<10 \hspace{-6pt}\else\hspace{-9pt}\fi % 根据数字设置间距
   \hfill % 左对齐
}
\makeatother

\let\oldbibitem\bibitem
\renewcommand{\bibitem}[1]{\oldbibitem{#1}\itemsep=-6pt} %设置item间距
\renewcommand{\labelenumi}{(\arabic{enumi})} %(1)

\def\X{$\times$}